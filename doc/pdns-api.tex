% Created 2014-06-19 Thu 17:51
\documentclass[a4paper]{article}
\usepackage[utf8]{inputenc}
\usepackage[T1]{fontenc}
\usepackage{fixltx2e}
\usepackage{graphicx}
\usepackage{longtable}
\usepackage{float}
\usepackage{wrapfig}
\usepackage{rotating}
\usepackage[normalem]{ulem}
\usepackage{amsmath}
\usepackage{textcomp}
\usepackage{marvosym}
\usepackage{wasysym}
\usepackage{amssymb}
\usepackage{hyperref}
\tolerance=1000
\usepackage{color}
\usepackage{listings}
\hypersetup{
colorlinks,%
citecolor=black,%
filecolor=black,%
linkcolor=blue,%
urlcolor=black
}
\author{Fyodor Yarochkin, V.B. Kropotov, Vitaly Chetvertakov}
\date{\today}
\title{PDNS -  API specification}
\hypersetup{
  pdfkeywords={},
  pdfsubject={},
  pdfcreator={Emacs 24.3.50.1 (Org mode 8.2.6)}}
\begin{document}

\maketitle
\tableofcontents


\section{Revision History}
\label{sec-1}
\subsection{Change Record}
\label{sec-1-1}
\begin{center}
\begin{tabular}{r|l|l|l}
\hline
Version & Date & Name & Revision Description\\
\hline
2.1 & \textit{<2014-03-11 Wed>} & Fyodor & initial revision.\\
\hline
\end{tabular}
\end{center}

\section{Introduction}
\label{sec-2}
PDNS is a system that indexes DNS queries data and performs some
DNS-based detection/classification of DNS content.

This document describes API of the system.
\section{Authentication}
\label{sec-3}
API is provided over HTTPs link at <pdns> url.
Each system user should access system using API key which is passed to
PDNS platform as additional HTTP POST or HTTP GET parameter.

\section{Query API}
\label{sec-4}
PDNS platform allows to query different types of data. Following are
the API endpoints that could be used to query variety of data
collected by PDNS platform:

\subsection{Query last detected domains/IPs}
\label{sec-4-1}
This query will return list of last domains, which were observed
 within last 60 seconds by DGA platform.
 Each domain entry will have source, DGA score, periodicity
score and query response.

The each of the entry in this list has 60 second TTL, after which the
entry will expire (and be deleted).

Each of the elements is deleted after query.

Query format:
\lstset{language=HTML,label= ,caption= ,numbers=none}
\begin{lstlisting}
POST /api/v1/lookup/last[/<query>]
   apikey=[apikey]

--optional--parameters
   callback_url=[url]
   score=<not less than score>
   ttl=<not older than ttl>
\end{lstlisting}


\subsection{Query clusters}
\label{sec-4-2}
This query returns array of clusters and detected suspicious domain
names.
\lstset{language=HTML,label= ,caption= ,numbers=none}
\begin{lstlisting}
POST /api/v1/lookup/clusters/[/<domain>]
   apikey=[apikey]

--optional--parameters
   callback_url=[url]
   score=<not less than score>
   ttl=<not older than ttl>
\end{lstlisting}

if query is specified, only clusters matching given <domain> name will
be returned.

\subsection{PDNS query}
\label{sec-4-3}
these queries support API similar to \url{https://api.dnsdb.info/}.

\subsubsection{rrset queries}
\label{sec-4-3-1}
\lstset{language=HTML,label= ,caption= ,numbers=none}
\begin{lstlisting}
POST /api/v1/lookup/rrset/name/OWNER_NAME/<RRTYPE>/<BAILIWICK>/
   apikey=[apikey]

--optional--parameters
   callback_url=[url]
   score=<not less than score>
   ttl=<not older than ttl>
\end{lstlisting}

\subsubsection{rdata queries}
\label{sec-4-3-2}
\lstset{language=HTML,label= ,caption= ,numbers=none}
\begin{lstlisting}
POST /api/v1/lookup/rdata/TYPE/VALUE/<RRTYPE>/
   apikey=[apikey]

--optional--parameters
   callback_url=[url]
   score=<not less than score>
   ttl=<not older than ttl>
\end{lstlisting}

\section{Interpreting Results}
\label{sec-5}

Results are given in JSON format. TBD

\section{Sample usage}
\label{sec-6}

\lstset{language=sh,label= ,caption= ,numbers=none}
\begin{lstlisting}
#!/bin/sh
curl --insecure --data "api_key=key&score=0.2" https://pdnsmachine.org/api/v1/check_last
\end{lstlisting}
% Emacs 24.3.50.1 (Org mode 8.2.6)
\end{document}